%\documentclass[lettersize,journal]{IEEEtran}
\documentclass[sigconf,screen,review]{acmart}
\usepackage[utf8]{inputenc}
\usepackage{outlines}
\usepackage{graphicx}
\usepackage[inline]{enumitem}
\usepackage{wrapfig}
\usepackage{makecell}
\usepackage{pifont}% http://ctan.org/pkg/pifont
\usepackage{amsthm}
\usepackage{amsmath}
\usepackage{mathtools}
%\usepackage{amssymb}
\usepackage{stmaryrd} % Additional math symbols.
\usepackage{amsfonts}
\usepackage{flushend}
\usepackage{balance}
%\usepackage[hyphens]{url}
\usepackage{hyperref}
\usepackage{multicol}
\usepackage{multirow}
\usepackage{setspace} % For double, 1.5, single spacing etc.
\usepackage{xspace}
\usepackage{listings}
\usepackage{ifthen}
\usepackage{verbatim}
\usepackage{float} % Defines newfloat used below
\usepackage{subcaption}
\usepackage{microtype}
\usepackage{mathptmx} % Fails the display of parentheses in math environment
\usepackage{textcomp}
%\usepackage[sort&compress]{natbib}
\usepackage{booktabs}
\usepackage{longtable}
\usepackage{blindtext}
\usepackage{fancyhdr}
%These two let you use multibyte utf-8 characters, e.g. 03bb - λ 
\usepackage[mathletters]{ucs}
\usepackage[utf8]{inputenc}
\usepackage{wasysym}	% Defines \cent, \currency, \brokenvert
\usepackage{tikz}
\usepackage{esvect}
\usepackage{csquotes}
\usepackage{array}
\usepackage{xcolor, colortbl}
\usepackage{tablefootnote}
\usepackage{supertabular}
% hyperref redefines a number of macros, so it should be last.  Empirically,
% doing so eliminates compiler warnings.
%\usepackage[bookmarks, colorlinks, citecolor=green, urlcolor=blue, 
%                    filecolor=blue, linkcolor=blue]{hyperref}

% According to the hyperref readme, algorithm must follow hyperref
\usepackage{algorithm}
\usepackage{algorithmicx}
\usepackage{algpseudocode}

% for \newcolumntype macro
 \newcolumntype{L}{>{$}l<{$}}
 	\newcolumntype{C}{>{$}c<{$}}

% required for combined latex/pdf xfig figures
\DeclareGraphicsRule{.pdftex}{pdf}{.pdftex}{}


%~~~~~~~~~~~~~~~~~~~~~~~~~~~~~~~~~~~~~~~~~~~~~~~~~~~~~~~~~~~~~~~~~~~~~~~~~~~~~~
% Macros																{{{1

% English
\newcommand{\cf}{\hbox{\emph{cf.}}\xspace}
\newcommand{\deletia}{\ldots [deletia] \ldots}
\newcommand{\etal}{\hbox{\emph{et al.}}\xspace}
\newcommand{\eg}{\hbox{\emph{e.g.}}\xspace}
\newcommand{\ie}{\hbox{\emph{i.e.}}\xspace}
\newcommand{\scil}{\hbox{\emph{sc.}}\xspace} %scilicet: it is permitted to know
\newcommand{\st}{\hbox{\emph{s.t.}}\xspace}
\newcommand{\wrt}{\hbox{\emph{w.r.t.}}\xspace}
\newcommand{\etc}{\hbox{\emph{etc.}}\xspace}
\newcommand{\viz}{\hbox{\emph{viz.}}\xspace} %videlicet: it is permitted to see


% % Algorithms
% \newfloat{Protocol}{thp}{lop}
% \DeclareMathOperator{\cbar}{||} %denotes concurrency in protocol floats.
% \newfloat{Program}{thp}{lop}
% \newfloat{Procedure}{thp}{lop}
% \providecommand*{\algorithmautorefname}{Algorithm}

%~~~~~~~~~~~~~~~~~~~~~~~~~~~~~~~~~~~~~~~~~~~~~~~~~~~~~~~~~~~~~~~~~~~~~~~~~~~~~~
% Theorems, etc.														{{{2
%\newenvironment{proof-idea}{\noindent{\bf Proof Idea}\hspace*{1em}}{\bigskip}

%\theoremstyle{plain}
%\newtheorem{thm}{Theorem}[section]
%\newtheorem{lem}[thm]{Lemma}
%\newtheorem{prop}[thm]{Proposition}
%\newtheorem{cor}[thm]{Corollary}

\theoremstyle{remark}
\newtheorem*{rem}{Remark}

\theoremstyle{definition}
\newtheorem{defn}{Definition}[section]
\providecommand*{\defnautorefname}{Definition}
%\newtheorem{conj}{Conjecture}
%}}}~~~~~~~~~~~~~~~~~~~~~~~~~~~~~~~~~~~~~~~~~~~~~~~~~~~~~~~~~~~~~~~~~~~~~~~~~~~


%~~~~~~~~~~~~~~~~~~~~~~~~~~~~~~~~~~~~~~~~~~~~~~~~~~~~~~~~~~~~~~~~~~~~~~~~~~~~~~
% Aliases																{{{1
\newcommand{\infinity}{\infty}
%}}}~~~~~~~~~~~~~~~~~~~~~~~~~~~~~~~~~~~~~~~~~~~~~~~~~~~~~~~~~~~~~~~~~~~~~~~~~~~

%~~~~~~~~~~~~~~~~~~~~~~~~~~~~~~~~~~~~~~~~~~~~~~~~~~~~~~~~~~~~~~~~~~~~~~~~~~~~~~
% Missing unicode chars, other brokenness in ucs/inputenc {{{1
\DeclareUnicodeCharacter{183}{\cdot}						% ·
\DeclareUnicodeCharacter{931}{\ensuremath\Sigma}			% Σ
\DeclareUnicodeCharacter{9001}{\ensuremath\langle}			% 〈
\DeclareUnicodeCharacter{9002}{\ensuremath\rangle}			% 〉
\DeclareUnicodeCharacter{9608}{\ensuremath\blacksquare}		% █
\DeclareUnicodeCharacter{1013}{\in}							% ϵ
\DeclareUnicodeCharacter{8213}{---}							% ―

\renewcommand{\textcent}{\cent}
\renewcommand{\textcurrency}{\currency}
\renewcommand{\textyen}{\yen}
\renewcommand{\textbrokenbar}{\brokenvert}
%}}}~~~~~~~~~~~~~~~~~~~~~~~~~~~~~~~~~~~~~~~~~~~~~~~~~~~~~~~~~~~~~~~~~~~~~~~~~~~

\usepackage{framed}
\newcounter{RQCounter}
\newcommand{\RQ}[1]{%
	\stepcounter{RQCounter}
	\begin{framed}%
		\noindent\textbf{Research Question \arabic{RQCounter}: }%
		#1\end{framed}
}

\newcommand{\cmark}{\ding{51}}%
\newcommand{\xmark}{\ding{55}}%

\DeclareTextFontCommand{\findings}{\normalfont\itshape\bfseries}

\newlength{\emstr}
\setlength{\emstr}{0.75em plus 1ex minus 1ex}
\newcommand{\boldpara}[1]{%
	\smallskip%
	\par\noindent\textbf{\textit{#1}}\hspace{\emstr}
}%

\def\sectionautorefname{Section}
\def\subsectionautorefname{Section}
\def\subsubsectionautorefname{Section}

% vim:foldmethod=marker

%%%%%%%%%%%%%%%%%%%%%%%%%%%%%%%%% Comments %%%%%%%%%%%%%%%%%%%%%%%%%%%%%
\newboolean{showcomments}
\setboolean{showcomments}{true} % comment this line to deactivate comments
\ifthenelse{\boolean{showcomments}}{
  \newcommand{\nbc}[3]{
    {\colorbox{#3}{\bfseries\sffamily\scriptsize\textcolor{white}{#1}}}%
    {\textcolor{#3}{\sf\small
    %$\blacktriangleright$
    \textit{#2}
    %$\blacktriangleleft$
    }}}
  \newcommand{\todo}[1]{\nbc{TODO}{#1}{blue}\xspace}
  \newcommand{\mar}[1]{\nbc{MAR}{#1}{cyan}\xspace}
  \newcommand{\earl}[1]{\nbc{EARL}{#1}{teal}\xspace}
  \newcommand{\fe}[1]{\nbc{FEDERICA}{#1}{violet}\xspace}
  \newcommand{\carlos}[1]{\nbc{CARLOS}{#1}{olive}\xspace}
}{
  \newcommand{\nbc}[3]{}
  \newcommand{\todo}[1]{}
  \newcommand{\mar}[1]{}
  \newcommand{\earl}[1]{}
  \newcommand{\fe}[1]{}
  \newcommand{\carlos}[1]{}
}


\newcommand{\ApproachName}{{\sc Imhotep}}
\newcommand{\simhotep}{$S_{Imhotep}$}
\newcommand{\timhotep}{$T_{Imhotep}$}
\newcommand{\CaseStudy}{Kromaia}

\acmBooktitle{Companion Proceedings of the 32nd ACM Symposium on the Foundations of Software Engineering (FSE '24), July 15--19, 2024, Porto de Galinhas, Brazil}

\begin{document}

\title{Automated Software Transplantation on Procedural Content Generation}
%\author{Mar Zamorano, Carlos Cetina, Federica Sarro}

\author{Mar Zamorano}
\email{maria.lopez.20@ucl.ac.uk}
\email{mzamorano@usj.es}
\affiliation{%
  \institution{University College London}
  \city{London}
  \country{UK}
}
\affiliation{%
  \institution{Universidad San Jorge}
  \city{Zaragoza}
  \country{Spain}
}

\author{Carlos Cetina}
\email{ccetina@usj.es}
\affiliation{%
  \institution{University College London}
  \city{London}
  \country{UK}
}
\affiliation{%
  \institution{Universidad San Jorge}
  \city{Zaragoza}
  \country{Spain}
}

\author{Federica Sarro}
\email{f.sarro@ucl.ac.uk}
\affiliation{%
  \institution{University College London}
  \city{London}
  \country{UK}
}

\renewcommand{\shortauthors}{Zamorano et al.}

\begin{abstract}
Procedural Content Generation (PCG) aims for the (semi) automatic generation of new content within video games. However, developers usually have limited control over the generation process.
We propose the first transplantation algorithm in the field of PCG that allows game developers to choose an organ from a donor and a host that will receive the organ. Through our approach, we aim to search for an appropriate solution to integrate the organ into the host. We also study two distinct objective functions: The one accepted in the literature of  software transplantation (test-based variant), and a novel objective function (simulation-based variant).
In our evaluation, we have transplanted a total of 129 distinct organs from the scenarios of \CaseStudy{} (a commercial video game) into 5 video game bosses as hosts.
Our simulation-based variant produces results that are 1.5 times superior to those of the test-based variant and 2.5 times superior to the PCG baseline. The statistical analysis confirms the significance of these differences and highlights the substantial magnitude of improvement.
Furthermore, a focus group with game developers indicated their satisfaction with the generated content.
Our analysis of the results also reveals organ interactions that have not been  identified previously in the literature.
\end{abstract}

% \begin{IEEEkeywords}
% Automated Software Transplantation, Auto-transplantation, Procedural Content Generation, Search-Based Software Engineering, Model-Driven Engineering
% \end{IEEEkeywords}

\keywords{Automated Software Transplantation, Auto-transplantation, Procedural Content Generation, Search-Based Software Engineering, Model-Driven Engineering}


\maketitle

\section{Introduction}

\IEEEPARstart{T}{he} video games industry grows significantly every year~\cite{rykala2020growth}. In 2019, the video games industry became the largest entertainment industry in terms of revenue after surpassing the combined revenues of the movie and music industries~\cite{politowski2021game}. In 2021, video games generated revenues of \$180.3 billion~\cite{wijman2021games}, and in 2022, the estimated revenues were of \$184.4 billion~\cite{wijman2022games}. Overall, the sum of revenues generated from 2020 to 2022 was almost \$43 billion higher than originally forecasted for the period. 
%In 2019, out of a total population of 18.9M software developers, 8.8M developers were involved in video games development at some point~\cite{devData}, and by the end of 2022~\cite{devNation}, those who self-identified as software developers accounted for 24\% of all professional video game developers, which is a higher percentage than those who self-identified as game designers (23\%), artists (15\%), UI designers (8\%), or QA engineers (5\%) - roles that are typically identified by consumers as core roles for video games development.

Video games are complex creations where art and software go hand in hand during the development process to conform the final product. Hence, development teams are conformed by different profiles, where the majority are software developers (24\%), but also include game designers (23\%), artists (15\%), UI designers (8\%), and QA engineers (5\%), based on a recent survey with professional game developers~\cite{devNation}. In a video game, software often permeates every aspect of the development, since it governs all the elements and actions that can appear or happen within the game. For instance, software controls the logic behind the actions of NPCs\footnote{Non-playable characters which are not controlled by a player} within the game (often through state machines or decision trees). As video games become more and more advanced, their software also becomes more complex.

To alleviate the complexity of video game development, most video games are developed using game engines. The most popular video game engines are Unity~\footnote{\url{https://unity.com/}} and Unreal~\footnote{https://www.unrealengine.com/}). Game engines are development environments that integrate a graphics engine and a physics engine as well as tools to accelerate development. For example, they provide a ready-to-use implementation of gravity or collisions between elements. Game engines significantly speed up the development of video games. For game developers, the main challenge is to develop the game content. Game content includes from the levels to the NPCs or game items such as weapons and power ups.

% --- PHARAGRAPH TO MOVE --- 
%Software for video games can be implemented either through code or through software models. Developing a game through code enhances the control that developers have over the game. The downside to developing games through code is dealing with the implementation details. On the other hand, models elevate the abstraction level using domain concepts that are closer to its developers. Developing a game through modeling relieves the developers from the implementation details, and allows non-technical roles (such as level designers and artists) to participate in key aspects of the development. This competitive advantage over traditional code is a potential reason behind the recent increase in the popularity of modeling for developing video games: one of the most widespread video game development engines (Unreal Engine provides a proprietary modeling language for development (Unreal Blueprints), and a recent survey in Model-Driven Game Development~\cite{zhu2019model} reveals that both UML and Domain Specific Language (DSL) models are also being adopted by development teams. The benefits of modeling present an opportunity for video game development teams to better cope with the growing complexity and demands of the software behind the games.

%The systematic reuse of previously generated content, or parts of it, in order to minimize bugs and cut development times is a very common occurrence within the video games industry~\cite{neto2009reuse}. However, manual reuse of content is typically frowned upon by the players: since developers tend to look at the components that they are most knowledgeable about for reuse, new content that is created from existing content tends to feel repetitive, unoriginal, and low-effort. This is especially true for video game content that has associated visual components (e.g. moving parts with reused animations).

%One of such demands is the demand for content. With every passing day, the demand for video games content keeps growing, with players requesting - and more often than not, expecting - more content than developers can produce. 
Content generation is a generally slow, tedious, costly, and error-prone manual process. In order to cope with the growing demand for content for video games, researchers are working towards Procedural Content Generation (PCG). PCG refers to the field of knowledge that aims at the (semi) automatic generation of new content within video games~\cite{hendrikx2013procedural}. Usually, current PCG approaches work as follows: developers provide initial content (usually human-generated content) into an algorithm to work with. Afterwards, the algorithm (Traditional, Machine Learning, or Search-Based methods) will generate new content. Only a few traditional methods have succeeded in providing tools used by the industry to randomly generate vegetation (e.g., SpeedTree in Unreal and Unity).

In this paper, we propose a new angle to tackle PCG for video games inspired by transplantation techniques~\cite{barr2015automated}, that we named Procedural Content Transplantation (PCT). In medicine, \textit{transplantation} is a procedure in which cells, tissues, or organs of an individual are replaced by those of another individual or the same person~\cite{FARSHBAFNADI2023599}. In software, researchers understand transplantation as a procedure in which a fragment (organ) of a software element (donor) is transferred into another software element (host)~\cite{barr2015automated}. Software transplantation has achieved success on different tasks: program repair~\cite{weimer2009automatically,sidiroglou2014automatic}, testing~\cite{zhang2017automated}, security~\cite{yang2017malware}, or functionality improvements~\cite{sidiroglou2017codecarboncopy}.

%Current PCG approaches work as follows: developers provide initial content (usually human-generated content) into an algorithm to work with. Afterwards, the algorithm (Traditional, Machine Learning, or Search-Based methods) will generate new content. Only a few traditional methods have succeeded in providing tools used by the industry to randomly generate vegetation (e.g., SpeedTree in Unreal and Unity). 
Our PCT proposal introduces the transplantation metaphor into PCG. In our approach, the developers of a game will select an organ (a fragment of video game content) from a donor (video game content) and a host (other video game content) that will receive the organ. The organ and the host will serve as inputs for our transplantation algorithm that will generate new content for the game combining the organ and the host. Our hypothesis is that our transplantation approach can release latent content that results of combining fragments of existing content. Furthermore, our transplantation approach provides more control to developers in comparison to industry approaches that are based on random generation, leading to results that are closer to developers' expectations.

%Our PCT algorithm works with Search-Based Software Engineering. Search-Based Software Engineering has a better shot at exploring large search spaces than humans. In the state-of-the-art of software transplantation, the search evolves through genetic operations (crossover and mutation) and is guided by a test objective function. In the field of Search-Based Procedural Content Generation (SBPCG), the common practice is to guide the search through a simulation objective function. Taking into account the differences between both fields of research, we chose to leverage tests and simulations separately to guide the transplantation algorithm, and then compare the results obtained by the two objective functions.

%Another difference between the state-of-the-art of software transplantation and SBPCG is the representation of the problem. Software transplantation works directly over the source code of a program, whereas due to the nature of video games, in SBPCG it is common to use software models to represent the content (more specifically, the links between the elements that conform the content) within the game. Since our approach aims to solve a problem in the field of SBPCG, we use software models for our representation.

Our approach, called Imhotep\footnote{Our approach is named after Imhotep, who is considered by many to have written the Edwin Smith Papyrus (the oldest known manual of surgery).}, relies on Search-based Software Engineering (SBSE) because SBSE has demonstrated success on software transplantation~\cite{barr2015automated}. In the literature, software transplantation approaches guide the search by Test-suites. The transplantation assessment is determined by the amount of test that a candidate solution is able to pass. Our work not only explores the use of Test-suite (Test-based Imhotep variant) but also we explore the use of video game simulations (Simulation-based Imhotep variant), to guide the search.  Our hypothesis is that it is possible to harness video games' NPCs to run simulations that provide data to asses the transplantations. Within video games, it is typical the find NPCs that serve as companions to the player, adversaries to defeat, or inhabitants of the virtual world. These NPCs have pre-programmed behaviours that could be used in game simulations. For instance, in a first-person shooter game (like the renowned Doom), NPCs explore the game levels in search of weapons and power-ups to engage in combat with other NPCs or the player.

We have carried out our evaluation over the \CaseStudy{} case study. \CaseStudy{} is a video game about flying and shooting with a spaceship in a three-dimensional space\footnote{See the official PlayStation trailer to learn more about \CaseStudy{}: \url{https://youtu.be/EhsejJBp8Go}}. The game was released on PC, PlayStation, and translated to eight different languages.
To evaluate Imhotep, 129 different organs extracted from the scenarios of \CaseStudy{} are transplanted into 5 of the video game bosses that act as hosts, generating new video game bosses in the process. In total, our approach works with 645 transplants. To the best of our knowledge, our work has more transplants than previous work in the literature, with a maximum of 327 successful transplants~\cite{reid2020optimising}.
%We assess the quality of each generated boss by computing the duration of a match between the generated boss and a simulated player, a measurement that stems from the literature~\cite{browne2010evolutionary}. 

The results of the two Imhotep variants (Test-based and Simulation-based Imhotep) and a PCG baseline from the literature [14] are compared against an oracle (provided by developers). The results show that, out of the three approaches (the two Imhotep variants and the baseline), the content generated through the simulation-based Imhotep variant obtains the closest results to the oracle for all the generation scenarios (32\% better than the PCG baseline). The test-based Imhotep variant obtains the second place (25\% better than the PCG baseline), with the baseline obtaining worse results than the other two in all scenarios. The generated bosses are a promising starting point: developers can either include them directly in the game, modify them to better suit their needs, or inspect them to find novelties from which they can create more original designs.

Our contributions can be summarized as follows:
\begin{itemize}
    \item[\textbf{1}] Novel application of Software Transplantation on Procedural Content Generation (PCT approach),
    \item[\textbf{2}] Software Transplantation of software models in the field of video games development, and
    \item[\textbf{3}] Comparison of two objective functions based on the trends in Software Transplantation and on the trends in PCG.
\end{itemize}

The rest of the paper is structured as follows: Section~\ref{sec:Background} provides the research framework for our work. Section~\ref{sec:Approach} describes our approach, depicting its usage for PCG. Section~\ref{sec:Evaluation} details the evaluation of our approach. Section~\ref{sec:Results} highlights the results of our research. Section~\ref{sec:Discussion} discusses the outcomes of the paper and future lines of work. Section~\ref{sec:Threats} outlines the threats to the validity of our work. Section~\ref{sec:Related} reviews the works related to this one. Finally, 
Section~\ref{sec:Conclusion} concludes the paper by summarizing the main contributions and results.
\section{Background} \label{sec:Background}

\subsection{Model-driven video game development}

Video games are pieces of software that, like any other software, need to be designed, developed, and maintained over time. However, there are some particularities of video games that make them differ from traditional software, such as the artistic component of the videogame, the complexity of the rendering pipelines, the heterogeneous nature of video game development teams, and the abstract nature of the final purpose of a video game: fun. 

Hence, video games present characteristics that differentiate their development and maintenance from the development and maintenance of classic software. Examples of these differences can be found in how video game developers must contribute to the implementation of different kinds of artifacts (e.g., shaders, meshes, or prefabs) or in the challenges they face when locating bugs or reusing code for the video game~\cite{pascarella2018video, chueca2023consolidation}.

Nowadays, most video games are developed by means of game engines. Game engines are development environments that integrate a graphics engine and a physics engine as well as tools for both to accelerate development. The most popular ones are Unity and Unreal Engine, but it is also possible for a studio to make its own specific engine (e.g., CryEngine~\footnote{\url{https://www.cryengine.com}}). 

One key artifact of game engines are software models. Unreal proposes its own modeling language (Unreal Blueprints), and a recent survey in Model-Driven Game Development~\cite{zhu2019model} reveals that UML and Domain Specific Language (DSL) models are also being adopted by development teams. Developers can use the software models to create video game content instead of using the traditional coding approach. While code allows for more control over the content, software models raise the abstraction level, promoting the use of domain terms and minimizing implementation and technological details. Through software models, developers are freed from a significant part of the implementation details of physics and graphics, and can focus on the content of the game itself (see Fig.~\ref{fig:architecture}).

\begin{figure}[h]
    \centering
    \includegraphics[width=0.45\textwidth]{Figures/fig_bg_OverviewArtifactsVG.pdf}
    \caption{Overview of video game artifacts.}
    \label{fig:architecture}
\end{figure}

\subsection{\CaseStudy{}}

The research presented in this paper is framed within the context of a commercial video game case study, \CaseStudy{}. In particular, our evaluation uses the bosses of the video game to evaluate the approach. Each level of \CaseStudy{} consists of a three-dimensional space where a player-controlled spaceship has to fly from a starting point to a target destination, reaching the goal before being destroyed. The gameplay experience involves exploring floating structures, avoiding asteroids, and finding items along the route, while basic enemies try to damage the spaceship by firing projectiles. If the player manages to reach the destination, the final boss corresponding to that level appears and must be defeated in order to complete the level. 

Bosses can be built either using C++ code or software models. The top part of Figure~\ref{fig:scenario} depicts a boss fight scenario where the player-controlled ship (item A in the figure) is battling The Serpent (item B in the figure), which is the final boss that must defeated in order to complete Level 1. The bottom part of the figure illustrates the two possible development approaches for the boss.

\begin{figure}[h]
    \centering
    \includegraphics[width=\columnwidth]{Figures/MM_Scenario.pdf}
    \caption{Model-Driven Development vs. Code-Centric Development in the context of \CaseStudy{}}
    \label{fig:scenario}
\end{figure}

Even though Figure~\ref{fig:scenario} shows excerpts of the implementation of The Serpent both in the form of software models and code, it is not necessary to implement the two simultaneously. Although developers can mix both technologies, developing different parts of the boss using one or the other indistinctly, they are also free to implement the content using software models exclusively or to do so purely via code. However, the heterogeneous nature of video game development teams - comprised majorly of programmers~\cite{devNation}, but also counting game designers, artists, UI designers, and QA engineers within their ranks - possibly favours the use of software models over code, since the higher abstraction level of the former (combined with their detachment from more technical implementation details) empowers less tech-focused roles to embrace a more active participation in development tasks. Furthermore, an experiment~\cite{domingo2020evaluating} confirmed that video game developers make fewer mistakes and are more efficient when working with the models than with the code.

Within the context of \CaseStudy{}, the elements of the game are created through software models, and more specifically, through the Shooter Definition Model Language (SDML). SDML is a DSL model for the video game domain that defines aspects that are included in video game entities: the anatomical structure (including their components, physical properties, and connections); the amount and distribution of vulnerable parts, weapons, and defenses; and the movement behaviours associated to the whole body or its parts. SDML has concepts such as hulls, links, weak points, weapons, and AI components, and allows for the development of any game element, such as bosses, enemies, or environmental elements. The models are created using SDML and interpreted at runtime to generate the corresponding game entities. In other words, software models created using SDML are translated into C++ objects at runtime using an interpreter integrated into the game engine~\cite{blasco2021evolutionary}. More information on the SDML model can be found in the following video presentation: \url{https://youtu.be/Vp3Zt4qXkoY}.


\section{Our Proposal: \ApproachName{} } 
\label{sec:Approach}
This section explains  how \ApproachName{}  makes use of evolutionary computation and software models to transplant organs within video games content in order to create new content (i.e., \CaseStudy bosses in our case study). To facilitate the comprehension, we also provide the reader with an example of transplantation for a simplified version of a  \CaseStudy  \sq{boss} inspired by the \sq{Serpent} boss shown in Figure.~\ref{fig:scenario} with letter B. Given the popularity of software models for video-game development (see Section \ref{sec:Background}),  we designed \ApproachName{}  to work with models. Although our running example uses the SDML models of \CaseStudy, our approach is generic and can be used with other modelling languages because it exploits the idea of boundaries between model elements.

Figure.~\ref{fig:approach} shows an overview of \ApproachName{}. At the top left there is the input to our approach, namely the organ to be transplanted from the donor and the host where the organ will be transplanted into. Afterwards, \ApproachName{} detects the points of the organ that allows the transplantation and the points where the organ can be inserted into the host. To initialize the population of the evolutionary algorithm, the organ is cloned and transplanted in a random point. Genetic operations generate potential solutions for transplantation, while the objective function asseses the quality of these solutions. This process of generating and assessing is repeated until a specific stop condition is met. When the evolutionary algorithm finishes the execution, we obtain a ranked list based on the given objective function of the best transplantations between organ and host. Next, we describe each step of \ApproachName{} in the following subsections.


\begin{figure}[tb]
    \centering
    \includegraphics[width=0.35\textwidth]{Figures/overview.png}
    \caption{Overview of  \ApproachName{}, our proposal for PCT.}
    \label{fig:approach}
\end{figure}


\subsection{Input selection}

\ApproachName{} allows the developers to identify a source model content (donor) with the organ that will be transplanted, and a target model content (host). 
In our running example we present a simplified version of the meta-model, and the corresponding concrete syntax of the model (see Figure.~\ref{fig:metamodel+syntax}) from  \CaseStudy{}.  In such model \sq{Hulls} serve as the structural framework that define the anatomical composition of the models. For example, the boss presented in Figure~\ref{fig:scenario} (identified as \sq{B}) has its body built by hulls. \sq{Weak points} are conceptual elements that possess the vulnerability to be harmed.
\sq{Weapons} are tangible items capable of causing harm through direct contact, such as discharging projectiles like bullets. Hulls, weak points, and weapons are attached between them through \sq{Links}.
%We use a graphical representation to help the comprehension of the reader, however the original metamodel does not work with a graphical model representation as it is not a requirement on every metamodel. The type of model will depend on the metamodel and models that developers decide on.

In our example, the source donor model is a simplified version of the original \CaseStudy{} \sq{boss} \sq{Serpent}. Figure ~\ref{fig:donor_host} (a) shows the graphical representation of the donor's model, differentiating each element of the model with a letter from A to S. It also shows with dashed lines the elements selected as organ (namely, H, I, J, K, N, O, P, Q). The host is a model of a regular enemy that could appear in \CaseStudy{}. Figure.~\ref{fig:donor_host} (b) shows the graphical representation of the host model.

\begin{figure}[tb]
	\centering
	\includegraphics[width=0.35\textwidth]{Figures/metamodel+syntax.png}
	\caption{Simplified meta-model and corresponding syntax.}
	\label{fig:metamodel+syntax}
\end{figure}

\begin{figure}[tb]
    \centering
    \includegraphics[width=0.30\textwidth]{Figures/donor+host.png}
    \caption{(a) Donor model (with selected organ in dashed lines) and (b) host model.}
    \label{fig:donor_host}
\end{figure}

% \begin{figure}[h]
%     \centering
%     \includegraphics[width=0.2\textwidth]{Figures/host.png}
%     \caption{Host}
%     \label{fig:host}
% \end{figure}

\subsection{Boundary detection}
To transplant an organ into a host we need to find a way to connect them. To this end we exploit the boundaries between the model elements of the organ and the host. A boundary is a connection point capable of connecting two distinct model elements within a model. The connection is restricted by the rules of the metamodel. In the simplified example in Figure.~\ref{fig:metamodel+syntax}, the Source and Target meta-relationships are the boundaries between the model elements of the models conforming to that metamodel. In other model languages, there will be other meta-relationships with other names that will be the boundaries.

\ApproachName{} automatically identifies the boundaries of the selected organ, and all the boundaries of the host. In our running example, the boundaries of the organ are the connection points between donor and host. The elements that connect with the rest of the donor are H, K, and Q. Figure~\ref{fig:org_bound} (a) shows the donor and the selected organ with its boundaries (which are b11 for the H element; b16 for the K element, and b25 for the Q element). While, the host boundaries are all the points where its model elements connect. Figure~\ref{fig:org_bound} (b) shows all the boundaries of the host of our running example: The host has a total of 19 boundaries identified by a tag from ba to bs.

\begin{figure}[tb]
    \centering
    \includegraphics[width=0.45\textwidth]{Figures/donor+host+boundaries.png}
    \caption{(a) Donor  and (b) host model boundaries. The boundary is represented by a crossed circle.}
    \label{fig:org_bound}
\end{figure}

% \begin{figure}[h]
%     \centering
%     \includegraphics[width=0.35\textwidth]{Figures/host_boundaries.png}
%     \caption{Host model boundaries. The boundary is represented by a circle crossed.}
%     \label{fig:host_bound}
% \end{figure}

\subsection{Boundary mapping}
In the boundary mapping step, \ApproachName{} determines a mapping between the boundaries of the organ and the host. For each boundary in the organ, \ApproachName{} considers all compatible boundaries of the host, including the possibility of not connecting the boundary to the host boundaries. The boundary compatibility is determined by the metamodel.

Table~\ref{tab:boundaries} shows a boundary mapping between the organ and the host of the running example. The boundary b11 is a boundary from a \sq{Link} from the model and according to the metamodel it can connect to any \sq{Hull}, \sq{Weapon}, and \sq{Weak Point}. The boundaries b16 and b25 are both \sq{Hulls} and they can connect with any \sq{Link}.

\begin{table}[tb]
\centering
\resizebox{0.20\textwidth}{!}{
\begin{tabular}{|c|ll|}
\hline
{Organ boundaries} & \multicolumn{2}{c|}{{ \begin{tabular}[c]{@{}c@{}}Host \\      boundaries\end{tabular}}} \\ \hline
& \multicolumn{1}{c|}{ba} & \multicolumn{1}{c|}{bm} \\ \cline{2-3} 
& \multicolumn{1}{c|}{bd} & \multicolumn{1}{c|}{bp} \\ \cline{2-3} 
& \multicolumn{1}{c|}{bg} & \multicolumn{1}{c|}{bs} \\ \cline{2-3} 
\multirow{-4}{*}{b11} 
& \multicolumn{1}{c|}{bj} & \multicolumn{1}{c|}{Not connected} \\ \hline
& \multicolumn{1}{c|}{bb} & \multicolumn{1}{c|}{bc} \\ \cline{2-3} 
& \multicolumn{1}{c|}{be} & \multicolumn{1}{c|}{bf} \\ \cline{2-3} 
& \multicolumn{1}{c|}{bh} & \multicolumn{1}{c|}{bi} \\ \cline{2-3} 
& \multicolumn{1}{c|}{bk} & \multicolumn{1}{c|}{bl} \\ \cline{2-3} 
& \multicolumn{1}{c|}{bn} & \multicolumn{1}{c|}{bo} \\ \cline{2-3} 
\multirow{-6}{*}{\begin{tabular}[c]{@{}c@{}}b16\\    \\ b25\end{tabular}} 
& \multicolumn{2}{c|}{Not connected} \\ \hline
\end{tabular}}
\caption{Mapping of compatible organ-host boundaries.}
\label{tab:boundaries}
\end{table}

\subsection{Initialize population}
In evolutionary algorithms, a population is a collection of possible solutions for a problem. The encoding is the problem representation that an algorithm is capable to understand. 

In our work, the encoding requires a binary vector that represents the organ in the donor, and the boundary mapping (see Figure.~\ref{fig:encoding}). In the binary vector, each element from the model is a position in the vector. If a position in the vector has a \sq{1}, it means that the element from the model is part of the organ. On the other hand, each boundary from the organ gets assigned a compatible boundary from the host.
%In our work, each individual represents a software model from the game. We use a similar encoding version of Blasco \etal~\cite{blasco2021evolutionary} that has been adapted to work with transplantations. The size of the encoding in the previous work was 64 and in this work its size is of 150.
The initial population of \ApproachName{} contains individuals composed by the host and the organ placed in a random position (\ie a random mapping between the organ boundaries and the compatible organ boundaries).

\begin{figure}[tb]
    \centering
    \includegraphics[width=0.35\textwidth]{Figures/encoding.png}
    \caption{Example of encoding.}
    \label{fig:encoding}
\end{figure}

\subsection{Genetic operators}
\ApproachName{} uses traditional genetic operators (namely, selection, crossover, and mutation) to generate new individuals (\ie candidate solutions). 
Specifically, we use the ranking selection, which ranks the individuals based on the objective function and retains the top ones in the current population. We use a single, random, cut-point crossover, which selects two parent solutions at random, and determines a cut point uniformly at random to split them into two sub-vectors. Then, the crossover creates two children solutions by combining the first part of the first parent with the second part of the second parent for the first child, and the first part of the second parent with the second part of the first parent for the second child. Finally, the new offspring is mutated by changing any value of the encoding uniformly at random with a certain probability. 
Figure~\ref{fig:candidates} shows an example of new individuals that could results from our running example. For simplicity, these individuals have unaltered organs, but illustrate different boundary mappings between organ and host.

\begin{figure}[tb]
    \centering
    \includegraphics[width=0.25\textwidth]{Figures/candidates.png}
    \caption{Example of individuals.}
    \label{fig:candidates}
\end{figure}

\subsection{Objective function}

Our work proposes to harness video games' NPCs to run simulations that provide data to assess the transplantations. The first thing that differentiates video games from traditional software is that the basic requirement of video games is \sq{fun}. \sq{Fun} is an abstract concept and the developers are in charge of interpreting it. In fact, different developers may have different interpretations. For some game developers, \sq{fun} is achieved with a difficult game that is very rewarding when progress is made (e.g., Dark Souls~\cite{darksouls}). While for other developers, \sq{fun} is achieved by effortlessly killing enemies (e.g., Dynasty Warriors~\cite{dynastywarriors}). Therefore, we argue that the developer intent is key for content generation.

Specifically, we propose to introduce the generated content (each individual in the population) into a simulation of the video game. The simulation produces a data trace of the events that have occurred. Using the data from the trace, we can check how well aligned are the events with the intention of the developers. In the running example, the simulation is a duel between a spaceship and a boss. The simulation generates data about the duel, such as the damage inflicted. The intention of the developers may be that the duel ends with the victory of the spaceship with a remaining life of less than 10\%.

Our proposal does not require ad hoc development of simulations. We propose that the simulations leverage mainly the NPCs (but also more video game elements, such as scenarios or items like weapons or powerups). NPCs are naturally developed during the development process of a video game. In other words, NPCs are integral components of  most video game genres such as First-Person Shooter (FPS), Real-Time Strategy (RTS), our racing games. We aim two goals with the aforementioned. On the one hand, it makes the use of simulations cheaper, i.e. it does not involve additional development costs, and secondly, it facilitates fidelity to the video game compared to ad hoc development. In the running example, during the simulation, the generated content is the boss, who can be accompanied by more NPCs acting as secondary enemies. Additionally, the spaceship that confronts the boss is an NPC representing an allied ship. Finally, the scenario, and items such as weapons or powerups also belong to the game itself.

In this work, the Simulation-based \ApproachName{} (\simhotep{}) assesses the transplants through a simulation of a game battle between the boss (Host') and a NPC spaceship. The information retrieved from the simulation is the data that the developers regard as relevant, using their domain knowledge. Hence, our approach takes into account  the percentage of simulated player victories ($F_{Victory}$) and the percentage of simulated player health left once the player wins a duel ($F_{Health}$).
The calculation of $F_{Victory}$ and $F_{Health}$ is performed in the same way as Blasco \etal~\cite{blasco2021evolutionary}, as described below:

$F_{Victory}$ is calculated as the difference between the number of human player victories ($V_{P}$) and the optimal number of victories (33\%, according to the developers of \CaseStudy{} and their criteria) ($V_{Optimal}$):
\begin{equation}
F_{Victory} = 1 -\frac{\mid V_{Optimal} - V_{P} \mid}{ V_{Optimal}}
\end{equation}

$F_{Health}$, which refers to completed duels that end in spaceship victories, is the average difference between the spaceship's health percentage once the duel is over ($\Theta_{P}$) and the optimal health level that the spaceship should have at that point ($\Theta_{Optimal}$, 20\%, according to the developers):
\begin{equation}
F_{Health} = 1 - \frac{\sum\limits_{d=1}^{V_{P}}\frac{\mid \Theta_{Optimal} - \Theta_{P} \mid}{ \Theta_{Optimal}}}{V_{P}}
\end{equation}

$F_{Overall}$ is an average objective value for a boss model that includes the objective criteria described above. $F_{Overall}$ also includes a validation part. The validation part is to avoid models with inconsistencies. The validity of the models is performed by a run-time interpreter that is part of the game. When the model is stated as non-valid the value of Validity will be 0. $F_{Overall}$ can assume a value in [0, 1] which is used to assess a boss model when our \ApproachName{} approach is applied to the \CaseStudy{} case study.

\begin{equation}
F_{Overall} = min \left ( Validity, \frac{\sum\limits_{i=1}^{N}F_{i}}{N} \right )
\end{equation}


% The objective function in \ApproachName{} assesses the quality of each individual as a model. First, as done in previous work that use \CaseStudy{}~\cite{blasco2021evolutionary}, the models pass through a validation process followed by a quantitative measurement. In our work we assess quantitatively the objective function by two means: Test-based and Simulation-based objective functions. We use two different objective functions due to the differences in the the state-of-the-art of software transplantation and PCG. The state-of-the-art in software transplantation mainly work with Test-based objective function, while the state-of-the-art in NPCs PCG work with Simulation-based objective function.

% The validation step before the Test-based or Simulation objective function is a requirement that \CaseStudy{} integrates in the game to avoid models with inconsistent data. The validity of the models is performed by a run-time interpreter that is part of the game. When the model is stated as non-valid the value of the objective function will be 0.0.

% The models that pass the validation process will then be assess by the Test-based and Simulation-based objective functions.
% For the Test-based objective function we ask the developers to provide the set of tests that they consider relevant to our work. The developers from \CaseStudy{} provided us with a total of 243 tests selected based on their domain knowledge. The objective value will be retrieved when each individual pass through the 243 tests, normalized in a scale of [0, 1]. An individual which passes the 243 tests will obtain 1.0, on the contrary if it does not pass any test it will obtain 0.0.

% On the other hand, the Simulation-based objective function as in Blasco \etal~\cite{blasco2021evolutionary} simulates an in game battle between the boss and a player. The information retrieved from the simulation is the data that the developers regard as relevant, using their domain knowledge. Hence, our approach takes into account the percentage of simulated player victories ($F_{Victory}$) and the percentage of simulated player health left once the player wins a duel ($F_{Health}$).
% The calculation of $F_{Victory}$ and $F_{Health}$ is performed in the same
% way as Blasco \etal~\cite{blasco2021evolutionary}:

% $F_{Victory}$ is calculated as the difference between the number of human player victories ($V_{P}$) and the optimal number of victories (33\%, according to the developers of \CaseStudy{} and their criteria) ($V_{Optimal}$):
% \begin{equation}
% F_{Victory} = 1 -\frac{\mid V_{Optimal} - V_{P} \mid}{ V_{Optimal}}
% \end{equation}

% $F_{Health}$, which refers to completed duels that end in human player victories, is the average difference between the human player's health percentage once the duel is over ($\Theta_{P}$) and the optimal health level that the player should have at that point ($\Theta_{Optimal}$, 20\%, according to the developers):
% \begin{equation}
% F_{Health} = 1 - \frac{\sum\limits_{d=1}^{V_{P}}\frac{\mid \Theta_{Optimal} - \Theta_{P} \mid}{ \Theta_{Optimal}}}{V_{P}}
% \end{equation}

% $F_{Overall}$ is an average fitness value for a boss model that includes the fitness criteria described above. $F_{Overall}$ can assume a value in [0, 1] which is used to assess a boss model when our \ApproachName{} approach is applied to the \CaseStudy{} case study.

% \begin{equation}
% F_{Overall} = min( Validity, \frac{\sum\limits_{i=1}^{N}F_{i}}{N} )
% \end{equation}

\section{Experimental Design} \label{sec:Evaluation}

In this section we explain how we evaluate the feasibility of automated transplantation in video games through \ApproachName{}. To do so, we run an experiment evaluating \ApproachName, with a measure from the literature, and we have conducted a preliminary evaluation with human developers.
Through this section, we will present the research questions that we aim to answer, the evaluation
process (including the measure quality for the solutions and baseline), and the implementation details.

\subsection{Research Questions}
We aim to answer the following research questions:
\RQ{What is the quality of the models generated by
Imhotep in contrast to the models from the oracle?}
\RQ{What is the quality of the models generated whith each variant of Imhotep (Simulation-based and Test-based)?}
\RQ{Is there a significant difference between a traditional PCG approach and a transplantation approach?}

\subsection{Planning and execution}

Figure~\ref{fig:evaluation} shows an overview of the evaluation process. The upper part of the figure shows the software models selected from the original video game content provided by developers from \CaseStudy{}, which are the inputs for the test cases. 

In the figure, the output of the test cases are a baseline and the two variants of our approach. We used the work by Gallota \etal~\cite{gallotta2022evolving} as PCG baseline. Gallota \etal presented a hybrid Evolutionary Algorithm for generating NPCs, more precisely spaceships. Their approach combine an L-system with Feasible Infeasible Two Population Algorithm. Gallota \etal were able to evolve spaceships that match some statistics of human-designed spaceships.
The two variants of our \ApproachName{} approach work as described in Section~\ref{sec:Approach} to form the transplanted models that are considered to be the most relevant transplantations.

The evaluate the output of the test cases we compare them with an oracle. The oracle is extracted from the original software models from the video game \CaseStudy{}. The oracle and the output pass through a simulation provided by the game developers of \CaseStudy{}, which simulates an in game battle between the boss and a player. From the simulation we extract the duration, which is a metric commonly used by the literature~\cite{browne2010evolutionary}.

{\bf Duration:} The duration of duels between players and boss units is expected to be around a certain optimal value. For the video game case study, through tests and questionnaires with players, the developers determined that concentration and engagement for an average boss reach their peak at approximately 10 minutes ($T_{Optimal}$), whereas the maximum accepted time was estimated to be 20 minutes ($2*T_{Optimal}$). Significant deviations from that reference value are good design-flaw indicators: short games are probably too easy; and duels that go on a lot longer than expected tend to make players lose interest. The criterion $Q_{Duration}$ is a measure of the average difference between the duration of each duel ($T_{d}$) and the desired, optimal duration ($T_{Optimal}$):
\begin{equation}
Q_{Duration} =  1 - \frac{\sum\limits_{d=1}^{Duels}\frac{\mid T_{Optimal} - T_{d} \mid}{T_{Optimal}}}{Duels} 
\end{equation}

\begin{figure}[h]
    \centering
    \includegraphics[width=0.45\textwidth]{Figures/evaluation_process.png}
    \caption{Overview of our evaluation process.}
    \label{fig:evaluation}
\end{figure}

\subsection{Implementation details}

For the evaluation we used 5 different host (Vermis, Teuthus, Argos, Orion, and Maia), which are original bosses from the video game \CaseStudy{}. As a donor we used the scenarios of the video game, and we collected 129 organs, that are elements from the scenario. Each organ was transplanted individually to each boss. Hence, we obtain a total of 645 new individuals (5*129). Each variant of \ApproachName{} provided a total of 645 new individuals as output, 645 new individuals from Simulation-based and 645 individuals from Test-based. In the case of the baseline, to make it fair, the approach was executed 129 times with each one of the 5 different hosts to obtain 645 new individuals.

In order to compare the baseline to the variants of \ApproachName{}, we chose the parameters shown in Table~\ref{tab:evaluation_parameters} to calibrate the evolutionary algorithm and the objective function. We established the stop condition at 2 minutes and 30 seconds, ensuring that the approaches run long enough to obtain the best solutions. Even though the population size is 100 individuals, we only present the best candidate in each run of the variants and the same with the baseline. We assume that the best candidate solutions are those with the highest objective function value.

The evaluation of \ApproachName{} and the baseline was done
using two Pcs with the following specifications; Intel Core i7-8750H, 16GB, 2.2GHz; and  2x Intel(R) Xeon(R) CPU X5660, 64GB, 2.80GHz.
The implementation uses the Java(TM) SE Runtime Environment (JDK 1.8), together with Java as the programming language. 
For purposes of replicability, the implementation source code and the data (software models and oracles) are publicly available at the following URL:
\todo{package of replicability}

\begin{table}[h]
    \centering
    \begin{tabular}{ll}
        \hline
        \bf{Parameter description}            & \bf{Value}  \\ \hline
        Stop Time                        & 2m 30s \\
        Population Size                  & 100    \\
        Number of parents                & 2      \\
        Number of offspring from parents & 2      \\
        Crossover probability            & 1      \\
        Mutation probability             & 1/150 \\ \hline
    \end{tabular}
    \caption{Parameter settings}
    \label{tab:evaluation_parameters}
    \end{table}

% \subsection{Quality measurements}
% \label{subsec:Measurements}

% In a recent research done by Browne et al., the experimentation with game users showed that the following criteria stand out as being the most important: Completion, Duration, Uncertainty, Killer Moves, Permanence, and Lead Change \cite{browne2010evolutionary}. Our evaluation measures these criteria with values in the interval [0,1].

% {\bf Completion (Viability):} A game against a boss unit should end with more conclusions (victories for either the player or the boss) than draws/ties. The criterion $Q_{Completion}$ calculates a ratio of conclusions over total duel count:
% \begin{equation}
% Q_{Completion} = \frac{Conclusions}{Duels}
% \end{equation}

%  {\bf Uncertainty (Quality):} In order to keep players engaged with a duel, neither the player nor the boss unit should get extremely close to victory or defeat too early before the duel is settled, with ($T_{d}$) being its duration. Therefore, a duel is considered to be more uncertain the longer the time until the player's or the boss unit's health levels reach a dangerous/critical status ($P_{d}$ and $B_{d}$, respectively). For each duel, $Q_{Uncertainty}$ measures the average deviation between the time at which it is detected that one of the contenders is on the verge of defeat and the time corresponding to the duration of the duel.
% \begin{equation}
% Q_{Uncertainty} =  1 - \frac{\sum\limits_{d=1}^{Duels}\frac{T_{d} - min\left ( P_{d}, B_{d} \right )}{T_{d}}}{Duels} 
% \end{equation}

% {\bf Killer Moves:}   $Q_{KMoves}$ measures the proportion of killer moves by any contender ($K$), taking into account the moves that are considered to be remarkable highlights ($H$) but that are less important than killer moves. In the video game case study, the developers considered that a highlight move happens when either the boss unit or the player experiences a decrease in health; killer moves are those that make the difference in health between the contenders reach 30\%.
% \begin{equation}
% Q_{KMoves} =  1 - \frac{\sum\limits_{d=1}^{Duels}\frac{K_{d}}{H_{d}}}{Duels} 
% \end{equation}

% {\bf Permanence:} Duels with a high permanence value are games in which the advantages given by significant actions or moves by one of the contenders are unlikely to be immediately reverted by the opponent in terms of dominance. In the video game case study, the developers considered every highlight move and killer move to be meaningful actions, with recovery moves ($R$) being those that quickly cancelled the advantages given by other previous killer or highlight moves. The criterion $Q_{Permanence}$ is measured as follows:
% \begin{equation}
% Q_{Permanence} =  1 - \frac{\sum\limits_{d=1}^{Duels}\frac{R_{d}}{H_{d}+K_{d}}}{Duels} 
% \end{equation}

% {\bf Lead Change:} The lack of lead changes indicates low dramatic value. In the video game case study, the lead is determined at any given moment by considering the contender with the highest health level. This criterion is measured taking into account those highlight or killer moves that cause the lead to change ($L$) during the course of a duel:
% \begin{equation}
% Q_{LChange} = \frac{\sum\limits_{d=1}^{Duels}\frac{L_{d}}{H_{d}+K_{d}}}{Duels} 
% \end{equation}

% $Q_{Overall}$ calculates an average quality value for a model, including all of the quality criterion studied:
% \begin{equation}
% Q_{Overall} = \frac{\sum\limits_{i=1}^{N}Q_{i}}{N}
% \end{equation}
\section{Results}
\label{sec:Results}

\begin{figure*}
    \centering
    \includegraphics[width=\textwidth]{Figures/Imhotep_with_legend_and_oracle_average-v4.pdf}
    \caption{Results}
    \label{fig:results}
\end{figure*}

Figure \ref{fig:results} shows the results of the evaluation execution of our approach when using the two objective functions (Simulation-Based and Test-Based) from \ApproachName{} and the PCG Baseline. The executions are grouped by each host (boss of \CaseStudy{}) that has been used in our experiment (Vermis, Teuthus, Argos, Orion, and Maia). The last column, with shaded background, shows the average of all the hosts for each objective function and the baseline. In addition, the oracle indicates the value obtained by the human-generated final boss models that were obtained from \CaseStudy{}. 

Each boxplot is generated from the results of each host obtained from the transplantation of each of the 5 hosts with each of the 129 organs. Therefore, each boxplot represents 645 values of a specific host-organ transplantation in a final boss model. Figure \ref{fig:results} shows in each column how the quality values obtained for each of the three strategies studied in our evaluation differ from the values for the models generated by the developers, which are represented by the horizontal red dashed lines that cross each host column. The boxplots that are closer to the horizontal lines are more similar in quality to the models produced by the developers. Additionally, the use of boxplots allows for the representation of the different results for the strategies used.

\todo{Analysis of the results. Simulation has the best results, test also better than baseline...}


\begin{table*}[t!]
    \caption{Mean Values and Standard Deviations}
    \centering
    \resizebox{\textwidth}{!}{%
    \begin{tabular}{llllllll}
    \toprule
    &\multicolumn{1}{c}{Vermis}
    &\multicolumn{1}{c}{Teuthus}
    &\multicolumn{1}{c}{Argos}
    &\multicolumn{1}{c}{Orion}
    &\multicolumn{1}{c}{Maia}
    &\multicolumn{1}{c}{Overall}
    \\ \midrule
    Simulation
    & 0.699 $\pm$ 0.105 & 0.607 $\pm$ 0.074 & 0.439 $\pm$ 0.093 & 0.488 $\pm$ 0.087 & 0.430 $\pm$ 0.121 & 0.533 $\pm$ 0.142       
    \\
    Test
    & 0.424 $\pm$ 0.130 & 0.463 $\pm$ 0.105 & 0.321 $\pm$ 0.069 & 0.314 $\pm$ 0.068 & 0.295 $\pm$ 0.093 & 0.363 $\pm$ 0.117        
    \\
    Baseline
    & 0.254 $\pm$ 0.033 & 0.195 $\pm$ 0.018 & 0.201 $\pm$ 0.018 & 0.329 $\pm$ 0.008 & 0.084 $\pm$ 0.018 & 0.213 $\pm$ 0.083
    \\ \midrule     
\end{tabular}
}

\label{tab:results}
\end{table*}

\begin{table*}[t!]
    \caption{Max and Min values.}
    \centering
    \resizebox{\textwidth}{!}{%
    \begin{tabular}{lllllllllllll}
    \toprule
    & \multicolumn{2}{c}{Vermis}
    & \multicolumn{2}{c}{Teuthus}
    & \multicolumn{2}{c}{Argos}
    & \multicolumn{2}{c}{Orion}
    & \multicolumn{2}{c}{Maia}
    & \multicolumn{2}{c}{Overall}
    \\ \midrule
    \multicolumn{1}{c}{} 
    & \multicolumn{1}{c}{Max} & \multicolumn{1}{c}{Min} 
    & \multicolumn{1}{c}{Max} & \multicolumn{1}{c}{Min} 
    & \multicolumn{1}{c}{Max} & \multicolumn{1}{c}{Min}
    & \multicolumn{1}{c}{Max} & \multicolumn{1}{c}{Min} 
    & \multicolumn{1}{c}{Max} & \multicolumn{1}{c}{Min} 
    & \multicolumn{1}{c}{Max} & \multicolumn{1}{c}{Min} 
    \\
    Simulation & 1.042 & 0.482   & 0.854 & 0.443   & 0.992 & 0.235   & 0.842 & 0.304   & 1.285 & 0.253   & 1.285 & 0.235                          
    \\
    Test & 0.866 & 0.123   & 0.850 & 0.273   & 0.510 & 0.170   & 0.633 & 0.115   & 0.667 & 0.117   & 0.866 & 0.115   
    \\
    Baseline & 0.323 & 0.157   & 0.257 & 0.158   & 0.257 & 0.157   & 0.355 & 0.307   & 0.141 & 0.063    & 0.355 & 0.063
    \\ \midrule                     
\end{tabular}
}

\label{tab:maxmin}
\end{table*}
\section{Discussion}
\label{sec:Discussion}

% The results presented in the previous section highlight that both variants of \ApproachName{} outperform the baseline. In particular, the variant of \ApproachName{} that works with a simulation-based fitness achieves better results than the rest of the evaluated approaches. The generated bosses are a suitable starting point: developers can either include them directly in the game, modify them to better suit their needs, or inspect them to look for novelties from which they can create more original designs. However, after a thorough inspection of the results, we believe that there is room for improvement within this promising line of work. To that extent, we identified a series of issues that are impacting the results, and also conducted a focus group to gather the stance of the developers with regard to the outcomes of our research.

To begin with, our work revolves around the transplantation of organs between two very different types of content in video games: scenarios and bosses. One may wonder why not transplanting organs between contents of the same type, such as between bosses. Technically, it should also be a smaller challenge to transplant organs among the same type of content due to the similarities and shared structures. However, video games put the focus on fun, which is many times achieved by avoiding repetition. Since the number of bosses is usually very limited in video games, transplanting between bosses could lead to repetition, hurting fun and creating negative play experiences for the players. In contrast, scenarios provide an abundant and promising source of organs that can withstand repetition, since it is frequent for a relevant portion of a scenario to not be explored by a player during a game: while players spend most of the time playing within scenarios, the focus of scenarios on completing goals combined with their sheer extension renders them difficult to explore in full. Hence, reusing between bosses and scenarios is more original and relevant for fun. 
%As future work, we will also explore conducting transplants between contents of different games.

Since transplanting an organ to a host contributes to generating new desirable content, one might consider performing more than one transplant on the same host to continue creating novel content. In its current state, our approach allows for only one organ to be transplanted at a time, but it should be possible to repeatedly transplant the same organ onto the same host, or to consider chains of transplants where desirable combinations of organs can be identified and transplanted in bulk into a host. However, upon analyzing the results, we have detected various interactions between organs that may help guide an approach that considered multiple transplants: 

\begin{itemize}
    
    \item \textbf{Organ dependencies} occur when an organ requires for another organ to be present in the host to work properly. For instance, a spike weapon must be mounted on a hull belonging to the body of a boss and cannot appear by itself. In other words, a spike weapon organ depends on the existence of a hull organ to be able to be included in the boss.
    
    \item \textbf{Organ incompatibilities} happen when an organ should not appear in the host under any circumstances. For instance, consider attaching a black hole organ to a hull belonging to the boss. The black hole organ destroys everything it touches, so it would instantly end the boss without triggering the end condition for the game, since the battle is considered as completed only when the player is the one responsible for ending the boss. This would actively block player progress, which is undesirable for the game.
    
    \item \textbf{Organ synergies} are found when the functionality of an organ benefits from the existence of another organ in the host. For instance, adding one or more weapons to a hull where a weak spot is located protects the boss from the player, building a more interesting challenge.
    
    \item \textbf{Organ discordances} take place when the functionality of an organ is hindered by the existence of another organ in the host. For instance, annexing a hull with a mobile arm to another hull with a laser may cause the laser beam to be intermittently blocked, decreasing its attack capabilities.
    
\end{itemize}

So far, the literature on software transplantation does not tackle or even identify interactions between organs. Studying these organ interactions is a promising line of work to advance the concept of  transplantation both in video games and in the general software domain.

Concerning the focus group (see the bottom part of Figure.~\ref{fig:evaluation}), we conducted a survey with two developers from Entalto\footnote{https://www.entaltostudios.com/} and two from Kraken Empire\footnote{https://www.krakenempire.com/}. All of them are seasoned video game developers who devote most of their working hours to the software behind the games. We openly asked about their content preferences, presenting them with generated content whose origin (that is to say, generated by either \ApproachName{} or by the baseline) was masked, and there was unanimous preference for \ApproachName{}-generated content. 

Furthermore, they indicated that they would use it as primary content for the game rather than secondary. Primary content is that which conforms an essential part of the experience of the players, while secondary content is that which does not directly affect the main experience but contributes to creating the atmosphere of the game (for instance, distant decoration). Until now, PCG works generated results used as secondary content. In that sense, the possibility of using generated content as primary content represents an advancement in PCG. Developers justify this choice by arguing that the content generated by \ApproachName{} aligns better with the vision of the game, whereas the baseline-generated content feels more random in purpose even when reusing content that was created within the context and vision of the game by the developers.

% Finally, it is important to highlight that the simulation is not developed ad-hoc for \ApproachName{}, but rather reuses NPCs developed for the game in the simulation. In the Kromaia case study, a game within the shooting genre, NPCs are the bosses the player faces upon the end of the scenarios. This strategy of NPC reuse could be applied to other video game genres which also use NPCs that could be used for simulation purposes, such as FPS games with bots, racing games with rival competitors, or strategy games with enemy generals commanding troops faced by the player.
\section{Threats to Validity}
\label{sec:Threats}

To acknowledge the threats to the validity of our work, we use the classification suggested by De Oliveira \etal~\cite{oliveira2011threats}.

\textbf{1. Conclusion Validity Threats.}
To minimize \textit{not accounting for random variation}, we have a total of 645 transplants for each host on each variation of Imhotep and the baseline. Also, each transplant has run for 2 minutes and 30 seconds.
In order to address the \textit{lack of good descriptive statistics}, we present the standard deviation, min-max range and a box-plot from the results of the experiments realized.
We tackled the \textit{lack of a meaningful comparison baseline} by comparing our two variants of Imhotep with a recent traditional PCG approach as baseline. 

%o [+] Lack of formal hypothesis and statistical tests: the comparison described in the former paragraph must be based on a formal hypothesis and must be evaluated by a proper statistical test. By proper, we mean a statistical test that adheres to the characteristics of the underlying data under evaluation, through parametric or nonparametric statistical inference procedures.

\textbf{2. Internal Validity Threats.}
To mitigate \textit{poor parameter settings} we have presented the parameters used in our experiment, and for the PCG baseline we have used the parameters presented by the original work.
We provide the source code and the artifacts used in our experiments to allow its reproduction as suggested to avoid the \textit{lack of discussion on code instrumentation}.
We handled the \textit{lack of real problem instances} by selecting a commercial video game as the case study for the evaluation. Likewise, the problem artifacts (donor, organs, and hosts) were directly obtained from the video game developers and the documentation itself. 


\textbf{3. Construct Validity Threats.}
To prevent the \textit{lack of assessing the validity of cost measures} threat, we made a fair comparison between the two variants of our approach and the baseline. Furthermore, we used duration as our metric for the evaluation, which is a metric adopted and \textit{validated} from the literature~\cite{browne2010evolutionary}.
To mitigate the \textit{lack of discussing the underlying model subjected to optimization}, we use the original SDML of the video game provided by the developers of \CaseStudy{}.


\textbf{4. External Validity Threats.}
To mitigate the \textit{generalization} threat, we designed our approach to be generic and applicable not only to our industrial case study but also for generating content in other different video games. To apply our approach in other video games, three main ingredients are required as in other SBSE approaches: encoding, operations, and fitness function. The crossover and mutation operations are extensively utilized. The encoding and the fitness function depend on the content to generate. Our approach should be replicated with other DSL and video games before assuring its generalization.
To avoid the \textit{lack of a clear object selection strategy} in our experiment, we have selected the instances from a commercial video game, which represents real-world instances.

%o [+] Lack of evaluations for instances of growing size and complexity:Software Engineering techniques are designed to handle systems and teams that may vary in size and complexity. For instance, while a given software project may have a few requirements, other may have thousands. Therefore, a SBSE approach must be evaluated across a breadth of problem instances, both varying in size and complexity, to provide an assessment on the limits of the new technique.
\section{Related work} 
\label{sec:Related}
In this section, we discuss work that (1) tackles automated software transplantation and (2) video game content generation.

%\subsection{Automated Software Transplantation}
\noindent \textbf{Automated Software Transplantation}
%On functionality transplantation, 
Miles \etal~\cite{miles2012situ} and Petke \etal~\cite{petke2014using} proposed the first approaches that transplant software code in a same program (assuming that different versions of the programs are considered a same program). 
This seminal work has inspired follow up research to perform Automated Software Transplantation between different programs~\cite{barr2015automated}, or even different programming languages~\cite{marginean2021automated}  and platforms~\cite{kwon2017cpr}, as summarised below.
%When transplanting within a same program, there is no need for adapters (\ie alterations in organ or host to adapt the organ to fit into the host).
Sidiroglou-Douskos \etal~\cite{sidiroglou2015horizontal} proposed a technique that divides the donor program by specific functionality, each piece is called a \sq{shard}. 
%The approach insert the shard into the host without modifications, that is, the work from Sidiroglou-Douskos does not use adapters either.
On the other hand, Maras \etal~\cite{maras2015towards} proposed a three step general approach, without implementing it, which applies feature localization to identify the organ; then code analysis and adaptation, and finally feature integration. Wang \etal~\cite{wang2016hunter} instead of using feature localization, takes as inputs the desired type signature of the organ and a natural language description of its functionality. With that, the approach called Hunter uses any existing code search engine to search for a method to transplant in a database of software repositories. 
%Further, Hunter generates adapter functions to transform the types from the desired type signature into the type signatures of the candidate functions.
Allamanis \etal's SMARTPASTE~\cite{allamanis2017smartpaste} takes the organ and replace variable names with holes, the approach using a deep neural network fills the holes. 
%Allamanis \etal~\cite{allamanis2017smartpaste} use Gated Graph Neural Networks~\cite{li2015gated} to predict the correct variable name in an expression.
Unlike Allamanis \etal, who puts holes into the organ, Lu \etal~\cite{lu2018program} introduced program slicing where the host is provided with a draft of the code with holes, or natural language comments. Similarly to Wang \etal~\cite{wang2016hunter} , program splicing looks into a database of programs to identify a relevant code to the current transplant task. 
%Finally, the approach selects the more suitable result found to fill the holes in the draft.
Barr \etal propose $\mu$SCALPEL~\cite{barr2015automated}, an automatic code transplant tool that uses genetic programming and testing to transplant code from one program to another. 
$\mu$SCALPEL uses test cases to define and maintain functionalities, small changes are made to the transplanted code, and code that does not aid in passing tests can be discarded, reducing the code to its minimal functioning form. 
Subsequently, Marginean \etal proposes $\tau$SCALPEL~\cite{marginean2021automated} to achieve the transplantation between different programs and programming languages. 
Kwon \etal propose CPR~\cite{kwon2017cpr} that transplants an entire program between different platforms. CPR realizes software transplantation by synthesizing a platform independent program from a platform dependent program. 
%To synthesis the platform independent program, CPR uses PIEtrace~\cite{kwon2013pietrace} to construct a set of trace programs, which captures the control flow path and the data dependencies observed during a concrete execution, and replaces all the platform dependencies with the concrete values that it observed during the concrete execution. Finally, CPR merges all these trace programs together to handle any input, by replacing the concrete values observed during the executions, with input variables. 
To the best of our knowledge our is the first proposal addressing automated software transplantation in the field of content generation for video games. Our proposal allows the transplantation between different types of content. We have demonstrated that in this context a simulation-based objective function yield superior outcomes compared to the test-based objective function that previously attained the most favourable results in traditional software engineering transplantation ($\mu$SCALPEL~\cite{marginean2021automated}).

%\subsection{Procedural Content Generation}
\noindent \textbf{Procedural Content Generation}
%Procedural Content Generation (PCG) refers to the automation or semi-automation of the generation of content in video games~\cite{hendrikx2013procedural}. The types of content generated by PCG are diverse, such as vegetation~\cite{mora2021flora}, sound~\cite{plans2012experience}, terrain~\cite{frade2009breeding}, Non-Playable Characters~\cite{viana2022illuminating}, dungeons~\cite{viana2019survey}, puzzles~\cite{de2019procedural}, and even the rules of a game~\cite{browne2008automatic}. PCG is a large field spanning many algorithms~\cite{yannakakis2018artificial}, which can be grouped in three main categories according to the survey of PCG techniques by Barriga et al.~\cite{Barriga2019}: Traditional methods~\cite{freiknecht2017survey} that generate content under a procedure without evaluation; Machine Learning methods (PCGML)~\cite{Summerville2018,liu2021deep,souchleris2023reinforcement} that train models to generate new content; and Search-Based methods (SBPCG)~\cite{hendrikx2013procedural,togelius2011search} that generate content through a search on a predefined space guided by a meta-heuristic using one or more objective functions. 
PCG refers to the automation or semi-automation of the generation of content in video games~\cite{hendrikx2013procedural}. %The types of content generated by PCG are diverse, such as vegetation~\cite{mora2021flora}, sound~\cite{plans2012experience}, terrain~\cite{frade2009breeding}, Non-Playable Characters~\cite{viana2022illuminating} or even the rules of a game~\cite{browne2008automatic}. 
PCG is a large field spanning many algorithms~\cite{yannakakis2018artificial}, which can be grouped in three main categories~\cite{Barriga2019}: Traditional methods~\cite{freiknecht2017survey} generating content under a procedure without evaluation; Machine Learning methods (PCGML)~\cite{Summerville2018,liu2021deep,souchleris2023reinforcement} that train models to generate new content; and Search-Based methods (SBPCG)~\cite{hendrikx2013procedural,togelius2011search} that generate content through a search on a predefined space guided by a meta-heuristic using one or more objective functions. 
%An interesting aspect of SBPCG is the objective function (or fitness function) that guides the search towards an optimal solution. SBPCG differentiates between three different types~\cite{togelius2011search}: direct, simulation, and interactive. Direct objective functions are those that are based on the available knowledge of developers (that is, the developers themselves participate in the assessment of the objective function). Direct objective functions can be either theory-driven (meaning that the opinion of the developers is directly leveraged) or data-driven (meaning that information about relevant parameters is extracted from artefacts like questionnaires or player models). Simulation objective functions replicate real situations to estimate the behaviour of real players. Work in this area focuses mainly on developing more human-like agents, bots, and AIs to be used as objective functions. Simulation objective functions can be static, where the simulator agent does not change during the simulation, or dynamic, where agents that learn during simulation are used. Finally, interactive objective functions are those that involve players in the composition of the objective function.In SBPCG, interactive objective functions can be either explicit, when players are outright asked for their opinions, or implicit, when the data is indirectly extracted or inferred from the observation of the actions of the players and the results of those actions.
Our work falls in the SBPCG category and it generates content of the NPC type. In the context of NPC generation using SBPCG, Ripamonti \etal~\cite{ripamonti2021dragon} developed a novel approach to generate monsters adapted to players, considering the monster with more death rate the preferred by the player. To evaluate the monsters, they recreated an environment with the main aspects from a MMORPG~\footnote{Massive Multiplayer Online Role-Playing Games} game. Pereira \etal~\cite{pereira2021procedural_enemies} and later extended by Viana \etal~\cite{viana2022illuminating} seek for generating enemies that meet a difficulty criteria. Pereira \etal and Viana \etal use the same research academic game in their experimental designs. Blasco \etal~\cite{blasco2021evolutionary} focusses on generating spaceship enemies that are comparable to the ones manually created by developers. To generate spaceships, Gallota \etal~\cite{gallotta2022evolving} used a combination of Lindenmayer systems~\cite{lindenmayer1968mathematical} and evolutionary algorithm. Gallota \etal as well as Blasco \etal use a commercial video game in their evaluation.
In the context of ML, to the best of our knowledge there is a gap in the generation of NPC. ML research focus on other aspects of video games, such AI~\cite{brocchini2022monster} or graphical aesthetics~\cite{li2020automatic}.
The motivation of our work comes from the limitations that we detected in previous work. Previous work focused on speeding up development time. However, the influence of the developers on the generated content was limited. The generated content depended on randomness resulting on generated content not aligned with the intention of the developers. As a result, the generated content was either not used or used as secondary content. 
Our work is the first approach that tackles automated software transplantation if the field of video games. Furthermore, our proposal allows the transplantation between different types of content. More precisely, in this work, we transplant organs from a scenarios to NPCs.

%Our previous work also generates NPCs using SBSE~\cite{blasco2021evolutionary}. Our previous work focused on using search to speed up development time. However, in our previous work the influence of the developers on the generated content was limited. The generated NPC depended on randomness resulting on generated NPCs not aligned with the intention of the developers. As a result, the generated content was either not used or used as secondary content. In fact, the limitations of our previous work were the inspiration for moving to transplantation. The transplant-based approach of this work keeps control in the hands of the developers (who choose the organ to transplant) and helps to explore the latent content that exists in the video game.

%, our approach transplant scenario elements into a NPC to obtain a novel version of the NPC. From the best of our knowledge we are the first work applying transplantation in PCG. Guarneri \etal~\cite{guarneri2013golem} or Norton \etal~\cite{norton2017monsters}  generate NPC monsters through an evolutionary algorithm with the aim of obtaining a diversity set of new monsters. With the same goal, 

%Our research introduces a fresh perspective on content generation through the use of transplantation, which sets it apart from traditional procedural content generation (PCG) methods. Transplantation enables the seamless integration of various content types, facilitating in our work the transplant of elements from scenarios to NPCs.

% \subsection{MDE and Game Software Engineering}

% One of the challenges in software development is the environment used, as each environment and programming languages has unique characteristics. Software models, and more precisely Model Driven Engineering, study how to alleviate this problem by approaching software development from a platform-independent perspective through models. Video game developers must deal with this challenge as well and has motivated the research that combine software models and the domain of video games. 

% The 2010 survey of Software Engineering Research for Computer Games~\cite{ampatzoglou2010software} identified only one work that applied Model-Driven Development to video games~\cite{reyno2009automatic}. That work coined the term ‘‘Model-Driven Game Development’’ and presented a first approach to 2D game prototyping through Model-Driven Development. Specifically, they used UML classes and state diagrams that were extended with stereotypes, and a model-to-code transformation to generate C++ code.

% More recent work presents work that intended to minimize errors, time, and cost in multi-platform video game development and maintenance~\cite{Nunez17,Nunez13,Usman17}, or suggest the use of business process models as the modelling language for video games~\cite{Solis15}.

% In the intersection between software models and evolutionary computation, Williams \etal~\cite{Williams11} use an evolutionary algorithm to search for desirable game character behaviours in a text-based video game that plays unattended combats and that outputs an outcome result. The character behaviour is defined using a Domain-Specific Language. The combats are managed internally and are only driven by behaviour parameters, without taking into account a spatial environment, real-time representation, or visual feedback (which takes into consideration the physical interaction of the characters, variation in the properties, etc.).

% Another work that focuses on the intersection between software models and evolutionary computation is Avida-MDE~\cite{Goldsby2008}, which generates state machines that describe the behaviour of one of the classes of a software system (Adaptive Flood Warning System case study). The resulting state machines comply with developer requirements (scenarios for
% adaptation). Instead of generating whole models, Avida-MDE extends already existing models (object models and state machines) with new state machines that support new scenarios. The work in Goldsby and Cheng \etal~\cite{Goldsby2008} does not report the size of the generated state machines; however, the ones shown in the paper are around 50 model elements, which is significantly smaller than the more than 1000 model elements of the models of a commercial video game such as Kromaia.

% The work mentioned above focus on generating new content from models, which differs with our proposal of using MDE to transplant model fragments between models.

% \subsection{Conclusion}

% Our work differs from previous work for various aspects. 
% To the best of our knowledge our is the first paper addressing automated software transplantation if the field of video games. Our proposal allows the transplantation between different types of content. More precisely, we transplant elements from a scenario to an NPC. The use of MDE separate the problem from the platform, and even the specific video game, as a same Domain Specific Language can be used in different video games.
\section{Conclusion}
\label{sec:Conclusion}

Procedural Content Generation (PCG) aims for the (semi) automatic generation of new content within video games.  Typically, current PCG methods are operated by developers providing initial content to an algorithm, which then generates additional content. However, the developers have limited control over the generation process, which results in the generated content being used as secondary content, or not been used at all.

In this study, we empower developers by introducing the transplantation metaphor into PCG for the first time. Our approach allows game developers to choose an organ from a donor and a host that will receive the organ. Through our approach, we aim to search for a suitable solution to integrate the organ into the host. To guide our search, we propose two distinct objective functions: one based on test case following conventional software transplantation method, and another novel objective function based on simulations, proposed here.

Our proposal has been empirically assessed by using the commercial video game \CaseStudy{}. To evaluate our approach, we have transplanted a total of 129 distinct organs from the scenarios of \CaseStudy{} into 5 video game bosses, which serve as hosts. This transplantation process has resulted in the creation of 1290 new video game bosses. We then compare the outcomes of our approach (the two variants) with a PCG baseline.

Our simulation-based variant produces results that are 1.5 times superior to those of the test-based variant and 2.5 times superior to the baseline. The statistical analysis confirms the significance of these differences and highlights the substantial magnitude of improvement.
Furthermore, a focus group with game developers indicated that first, they would use the generated content by our approach, and secondly, that they would use it as primary content for the game rather than secondary.

Our results demonstrate that we have successfully generated new content through transplantation. Not only that alone, we have accomplish 645 transplantation in total for a commercial video game. Furthermore, our work achieves transplantation between different types of content which results in expanding the library of organs available. This can inspire researchers and developers to explore the use of different types of content for creating new content automatically. 
In addition, we have presented a novel objective function to guide search software transplantation, which has obtained better results than traditional one. This novel search guidance opens a door in the field of software transplantation.



\begin{acks}
    Work supported by National R+D+i Plan PID2021-128695OB-100, José Castillejo CAS18/00272, ERC grant 741278.
\end{acks}

\bibliographystyle{ACM-Reference-Format}
%\bibliographystyle{IEEEtran}
\bibliography{references}

\end{document}
