\section{Threats to Validity}
\label{sec:Threats}

To tackle possible threats to the validity of our work, we follow the classification suggested by De Oliveira \etal~\cite{oliveira2011threats}.

 \textbf{Conclusion Validity.}
To minimize \textit{not accounting for random variation}, we run each of the approach (\ie \simhotep, \timhotep and SBPCG) 30 times. Also, we make sure to assess the same number of solutions (\ie 645 new bosses) for each of the approaches, so to make the comparison fair. In order to address the \textit{lack of good descriptive statistics}, we present the standard deviation and a box-plot of the results. We also applied statistical significance tests (Mann-Whitney U) and effect size measurements ($\hat{A}_{12}$) following accepted guidelines~\cite{arcuri2013parameter}. We tackled the \textit{lack of a meaningful comparison baseline} by comparing \ApproachName{} to a recent and most relevant Search-Based PCG approach as a benchmark, as detailed in Section \ref{fig:evaluation}. 

\noindent \textbf{Internal Validity.}
%To mitigate \textit{poor parameter settings} we have presented the parameters used in our experiment, and for the PCG baseline we have used the parameters presented by the original work.
We provide the source code and the artefacts used in our experiments to allow for reproduction and replication, as suggested to avoid the \textit{lack of discussion on code instrumentation}.
We handled the \textit{lack of real problem instances} by using a commercial video game as the case study for our evaluation and by working closely with its developers in a real-world industrial setting. 
Likewise, the problem artefacts (donor, organs and hosts) were directly obtained from the video game developers and the documentation itself. 

\noindent \textbf{Construct Validity.}
To prevent the \textit{lack of assessing the validity of cost measures}, we made a fair comparison between the two variants of our approach and the SBPCG benchmark. Furthermore, we used a metric for the evaluation that has been widely adopted and \textit{validated} by the research community~\cite{browne2010evolutionary}.
%To mitigate the \textit{lack of discussing the underlying model subjected to optimization}, we use the original SDML of the video game provided by the developers of \CaseStudy{}.

\noindent \textbf{External Validity.}
To mitigate the lack of \textit{generalization} threat, we designed our approach to be generic and applicable not only to our industrial case study but also for generating content in other different video games. 
To avoid the \textit{lack of a clear object selection strategy} in our experiment, we have selected the instances from a commercial video game, which represents real-world instances.
In fact, \ApproachName{} can be applied where NPCs are available. NPCs are usually available in popular game genres such as car games (rival drivers), FPS games (bots), or RTS games (rival generals). For those cases were there is no NPC, the developers should ponder the trade-off of the cost of developing the NPCs and the benefits of generating content with our approach. Our approach should be replicated with other video games before assuring its generalization.


%o [+] Lack of evaluations for instances of growing size and complexity:Software Engineering techniques are designed to handle systems and teams that may vary in size and complexity. For instance, while a given software project may have a few requirements, other may have thousands. Therefore, a SBSE approach must be evaluated across a breadth of problem instances, both varying in size and complexity, to provide an assessment on the limits of the new technique.