\section{Conclusion}
\label{sec:Conclusion}

Procedural Content Generation (PCG) aims for the (semi) automatic generation of new content within video games.  Typically, current PCG methods are operated by developers providing initial content to an algorithm, which then generates additional content. However, the developers have limited control over the generation process, which results in the generated content being used as secondary content, or not been used at all.

In this study, we empower developers by introducing the transplantation metaphor into PCG for the first time. Our approach allows game developers to choose an organ from a donor and a host that will receive the organ. Through our approach, we aim to search for a suitable solution to integrate the organ into the host. To guide our search, we propose two distinct objective functions: one based on test case following conventional software transplantation method, and another novel objective function based on simulations, proposed here.

Our proposal has been empirically assessed by using the commercial video game \CaseStudy{}. To evaluate our approach, we have transplanted a total of 129 distinct organs from the scenarios of \CaseStudy{} into 5 video game bosses, which serve as hosts. This transplantation process has resulted in the creation of 1290 new video game bosses. We then compare the outcomes of our approach (the two variants) with a PCG baseline.

Our \simhotep{} produces results that are 1.5 times superior to those of the \timhotep{} and 2.5 times superior to the baseline. The statistical analysis confirms the significance of these differences and highlights the substantial magnitude of improvement.
Furthermore, a focus group with game developers indicated that first, they would use the generated content by our approach, and secondly, that they would use it as primary content for the game rather than secondary.

Our results demonstrate that we have successfully generated new content through transplantation. Not only that alone, we have accomplish 645 transplantation in total for a commercial video game. Furthermore, our work achieves transplantation between different types of content which results in expanding the library of organs available. This can inspire researchers and developers to explore the use of different types of content for creating new content automatically. 
In addition, we have presented a novel objective function to guide search software transplantation, which has obtained better results than traditional one. This novel search guidance opens a door in the field of software transplantation.

