\section{Discussion}
\label{sec:Discussion}

% The results presented in the previous section highlight that both variants of \ApproachName{} outperform the baseline. In particular, the variant of \ApproachName{} that works with a simulation-based fitness achieves better results than the rest of the evaluated approaches. The generated bosses are a suitable starting point: developers can either include them directly in the game, modify them to better suit their needs, or inspect them to look for novelties from which they can create more original designs. However, after a thorough inspection of the results, we believe that there is room for improvement within this promising line of work. To that extent, we identified a series of issues that are impacting the results, and also conducted a focus group to gather the stance of the developers with regard to the outcomes of our research.

To begin with, our work revolves around the transplantation of organs between two very different types of content in video games: scenarios and bosses. One may wonder why not transplanting organs between contents of the same type, such as between bosses. Technically, it should also be a smaller challenge to transplant organs among the same type of content due to the similarities and shared structures. However, video games put the focus on fun, which is many times achieved by avoiding repetition. Since the number of bosses is usually very limited in video games, transplanting between bosses could lead to repetition, hurting fun and creating negative play experiences for the players. In contrast, scenarios provide an abundant and promising source of organs that can withstand repetition, since it is frequent for a relevant portion of a scenario to not be explored by a player during a game: while players spend most of the time playing within scenarios, the focus of scenarios on completing goals combined with their sheer extension renders them difficult to explore in full. Hence, reusing between bosses and scenarios is more original and relevant for fun. 
%As future work, we will also explore conducting transplants between contents of different games.

Since transplanting an organ to a host contributes to generating new desirable content, one might consider performing more than one transplant on the same host to continue creating novel content. In its current state, our approach allows for only one organ to be transplanted at a time, but it should be possible to repeatedly transplant the same organ onto the same host, or to consider chains of transplants where desirable combinations of organs can be identified and transplanted in bulk into a host. However, upon analyzing the results, we have detected various interactions between organs that may help guide an approach that considered multiple transplants: 

\begin{itemize}
    
    \item \textbf{Organ dependencies} occur when an organ requires for another organ to be present in the host to work properly. For instance, a spike weapon must be mounted on a hull belonging to the body of a boss and cannot appear by itself. In other words, a spike weapon organ depends on the existence of a hull organ to be able to be included in the boss.
    
    \item \textbf{Organ incompatibilities} happen when an organ should not appear in the host under any circumstances. For instance, consider attaching a black hole organ to a hull belonging to the boss. The black hole organ destroys everything it touches, so it would instantly end the boss without triggering the end condition for the game, since the battle is considered as completed only when the player is the one responsible for ending the boss. This would actively block player progress, which is undesirable for the game.
    
    \item \textbf{Organ synergies} are found when the functionality of an organ benefits from the existence of another organ in the host. For instance, adding one or more weapons to a hull where a weak spot is located protects the boss from the player, building a more interesting challenge.
    
    \item \textbf{Organ discordances} take place when the functionality of an organ is hindered by the existence of another organ in the host. For instance, annexing a hull with a mobile arm to another hull with a laser may cause the laser beam to be intermittently blocked, decreasing its attack capabilities.
    
\end{itemize}

So far, the literature on software transplantation does not tackle or even identify interactions between organs. Studying these organ interactions is a line of work to advance the concept of  transplantation both in video games and in the general software domain.

Concerning the focus group (see the bottom part of Figure.~\ref{fig:evaluation}), we conducted a survey with two developers from Entalto~\cite{entaltoweb} and two from Kraken Empire~\cite{krakenweb}. All of them are seasoned video game developers who devote most of their working hours to the software behind the games. We openly asked about their content preferences, presenting them with generated content whose origin (that is to say, generated by either \ApproachName{} or by the baseline) was masked, and there was unanimous preference for \ApproachName{}-generated content. 

Furthermore, they indicated that they would use it as primary content for the game rather than secondary. Primary content is that which conforms an essential part of the experience of the players, while secondary content is that which does not directly affect the main experience but contributes to creating the atmosphere of the game (for instance, distant decoration). Until now, PCG works generated results used as secondary content. In that sense, the possibility of using generated content as primary content represents an advancement in PCG. Developers justify this choice by arguing that the content generated by \ApproachName{} aligns better with the vision of the game, whereas the baseline-generated content feels more random in purpose even when reusing content that was created within the context and vision of the game by the developers.

% Finally, it is important to highlight that the simulation is not developed ad-hoc for \ApproachName{}, but rather reuses NPCs developed for the game in the simulation. In the Kromaia case study, a game within the shooting genre, NPCs are the bosses the player faces upon the end of the scenarios. This strategy of NPC reuse could be applied to other video game genres which also use NPCs that could be used for simulation purposes, such as FPS games with bots, racing games with rival competitors, or strategy games with enemy generals commanding troops faced by the player.