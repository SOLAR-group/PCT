\section{Evaluation} \label{sec:Evaluation}

In this section we explain how we evaluate the feasibility of automated transplantation in video games through our \ApproachName{} approach. Our experiment compares \ApproachName{} with the two different objective functions, Test-based and Simulation-based, and a PCG baseline. 

The PCG baseline that we use is... \todo{explain the baseline}

We transplant on 5 different host, original bosses from the video game \CaseStudy{}. We have 129 donors, that are elements from the scenario of the game, called "Totems". We obtain a total of 645 new host' from Simulation-based and 645 new host' from Test-based. \todo{for the baseline we force 129 repetitions per initial boss, so we obtain 645 new host as well}

The execution time for each transplantation is 2 minutes and 30 seconds.

We run the experiments in two laptops with the following specifications; Intel Core i7-8750H, 16GB, 2.2GHz; and  2x Intel(R) Xeon(R) CPU X5660, 64GB, 2.80GHz.

We evaluate each host' with the following quality measures:

\subsection{Quality measurements}
\label{subsec:Measurements}

In a recent research done by Browne et al., the experimentation with game users showed that the following criteria stand out as being the most important: Completion, Duration, Uncertainty, Killer Moves, Permanence, and Lead Change \cite{browne2010evolutionary}. Our evaluation measures these criteria with values in the interval [0,1].

{\bf Completion (Viability):} A game against a boss unit should end with more conclusions (victories for either the player or the boss) than draws/ties. The criterion $Q_{Completion}$ calculates a ratio of conclusions over total duel count:
\begin{equation}
Q_{Completion} = \frac{Conclusions}{Duels}
\end{equation}

{\bf Duration (Viability):} The duration of duels between players and boss units is expected to be around a certain optimal value. For the video game case study, through tests and questionnaires with players, the developers determined that concentration and engagement for an average boss reach their peak at approximately 10 minutes ($T_{Optimal}$), whereas the maximum accepted time was estimated to be 20 minutes ($2*T_{Optimal}$). Significant deviations from that reference value are good design-flaw indicators: short games are probably too easy; and duels that go on a lot longer than expected tend to make players lose interest. The criterion $Q_{Duration}$ is a measure of the average difference between the duration of each duel ($T_{d}$) and the desired, optimal duration ($T_{Optimal}$):
\begin{equation}
Q_{Duration} =  1 - \frac{\sum\limits_{d=1}^{Duels}\frac{\mid T_{Optimal} - T_{d} \mid}{T_{Optimal}}}{Duels} 
\end{equation}


 {\bf Uncertainty (Quality):} In order to keep players engaged with a duel, neither the player nor the boss unit should get extremely close to victory or defeat too early before the duel is settled, with ($T_{d}$) being its duration. Therefore, a duel is considered to be more uncertain the longer the time until the player's or the boss unit's health levels reach a dangerous/critical status ($P_{d}$ and $B_{d}$, respectively). For each duel, $Q_{Uncertainty}$ measures the average deviation between the time at which it is detected that one of the contenders is on the verge of defeat and the time corresponding to the duration of the duel.
\begin{equation}
Q_{Uncertainty} =  1 - \frac{\sum\limits_{d=1}^{Duels}\frac{T_{d} - min\left ( P_{d}, B_{d} \right )}{T_{d}}}{Duels} 
\end{equation}

{\bf Killer Moves:}   $Q_{KMoves}$ measures the proportion of killer moves by any contender ($K$), taking into account the moves that are considered to be remarkable highlights ($H$) but that are less important than killer moves. In the video game case study, the developers considered that a highlight move happens when either the boss unit or the player experiences a decrease in health; killer moves are those that make the difference in health between the contenders reach 30\%.
\begin{equation}
Q_{KMoves} =  1 - \frac{\sum\limits_{d=1}^{Duels}\frac{K_{d}}{H_{d}}}{Duels} 
\end{equation}

{\bf Permanence:} Duels with a high permanence value are games in which the advantages given by significant actions or moves by one of the contenders are unlikely to be immediately reverted by the opponent in terms of dominance. In the video game case study, the developers considered every highlight move and killer move to be meaningful actions, with recovery moves ($R$) being those that quickly cancelled the advantages given by other previous killer or highlight moves. The criterion $Q_{Permanence}$ is measured as follows:
\begin{equation}
Q_{Permanence} =  1 - \frac{\sum\limits_{d=1}^{Duels}\frac{R_{d}}{H_{d}+K_{d}}}{Duels} 
\end{equation}

{\bf Lead Change:} The lack of lead changes indicates low dramatic value. In the video game case study, the lead is determined at any given moment by considering the contender with the highest health level. This criterion is measured taking into account those highlight or killer moves that cause the lead to change ($L$) during the course of a duel:
\begin{equation}
Q_{LChange} = \frac{\sum\limits_{d=1}^{Duels}\frac{L_{d}}{H_{d}+K_{d}}}{Duels} 
\end{equation}

$Q_{Overall}$ calculates an average quality value for a model, including all of the quality criterion studied:
\begin{equation}
Q_{Overall} = \frac{\sum\limits_{i=1}^{N}Q_{i}}{N}
\end{equation}