\usepackage{graphicx}
\usepackage[inline]{enumitem}
\usepackage{wrapfig}
\usepackage{makecell}
\usepackage{pifont}% http://ctan.org/pkg/pifont
\usepackage{amsthm}
\usepackage{amsmath}
\usepackage{mathtools}
%\usepackage{amssymb}
\usepackage{stmaryrd} % Additional math symbols.
\usepackage{amsfonts}
\usepackage{flushend}
\usepackage{balance}
%\usepackage[hyphens]{url}
\usepackage{hyperref}
\usepackage{multicol}
\usepackage{multirow}
\usepackage{setspace} % For double, 1.5, single spacing etc.
\usepackage{xspace}
\usepackage{listings}
\usepackage{ifthen}
\usepackage{verbatim}
\usepackage{float} % Defines newfloat used below
\usepackage{subcaption}
\usepackage{microtype}
\usepackage{mathptmx} % Fails the display of parentheses in math environment
\usepackage{textcomp}
%\usepackage[sort&compress]{natbib}
\usepackage{booktabs}
\usepackage{longtable}
\usepackage{blindtext}
\usepackage{fancyhdr}
%These two let you use multibyte utf-8 characters, e.g. 03bb - λ 
\usepackage[mathletters]{ucs}
\usepackage[utf8]{inputenc}
\usepackage{wasysym}	% Defines \cent, \currency, \brokenvert
\usepackage{tikz}
\usepackage{esvect}
\usepackage{csquotes}
\usepackage{array}
\usepackage{xcolor, colortbl}
\usepackage{tablefootnote}
\usepackage{supertabular}
% hyperref redefines a number of macros, so it should be last.  Empirically,
% doing so eliminates compiler warnings.
%\usepackage[bookmarks, colorlinks, citecolor=green, urlcolor=blue, 
%                    filecolor=blue, linkcolor=blue]{hyperref}

% According to the hyperref readme, algorithm must follow hyperref
\usepackage{algorithm}
\usepackage{algorithmicx}
\usepackage{algpseudocode}

% for \newcolumntype macro
 \newcolumntype{L}{>{$}l<{$}}
 	\newcolumntype{C}{>{$}c<{$}}

% required for combined latex/pdf xfig figures
\DeclareGraphicsRule{.pdftex}{pdf}{.pdftex}{}


%~~~~~~~~~~~~~~~~~~~~~~~~~~~~~~~~~~~~~~~~~~~~~~~~~~~~~~~~~~~~~~~~~~~~~~~~~~~~~~
% Macros																{{{1

% English
\newcommand{\cf}{\hbox{\emph{cf.}}\xspace}
\newcommand{\deletia}{\ldots [deletia] \ldots}
\newcommand{\etal}{\hbox{\emph{et al.}}\xspace}
\newcommand{\eg}{\hbox{\emph{e.g.}}\xspace}
\newcommand{\ie}{\hbox{\emph{i.e.}}\xspace}
\newcommand{\scil}{\hbox{\emph{sc.}}\xspace} %scilicet: it is permitted to know
\newcommand{\st}{\hbox{\emph{s.t.}}\xspace}
\newcommand{\wrt}{\hbox{\emph{w.r.t.}}\xspace}
\newcommand{\etc}{\hbox{\emph{etc.}}\xspace}
\newcommand{\viz}{\hbox{\emph{viz.}}\xspace} %videlicet: it is permitted to see


% % Algorithms
% \newfloat{Protocol}{thp}{lop}
% \DeclareMathOperator{\cbar}{||} %denotes concurrency in protocol floats.
% \newfloat{Program}{thp}{lop}
% \newfloat{Procedure}{thp}{lop}
% \providecommand*{\algorithmautorefname}{Algorithm}

%~~~~~~~~~~~~~~~~~~~~~~~~~~~~~~~~~~~~~~~~~~~~~~~~~~~~~~~~~~~~~~~~~~~~~~~~~~~~~~
% Theorems, etc.														{{{2
%\newenvironment{proof-idea}{\noindent{\bf Proof Idea}\hspace*{1em}}{\bigskip}

%\theoremstyle{plain}
%\newtheorem{thm}{Theorem}[section]
%\newtheorem{lem}[thm]{Lemma}
%\newtheorem{prop}[thm]{Proposition}
%\newtheorem{cor}[thm]{Corollary}

\theoremstyle{remark}
\newtheorem*{rem}{Remark}

\theoremstyle{definition}
\newtheorem{defn}{Definition}[section]
\providecommand*{\defnautorefname}{Definition}
%\newtheorem{conj}{Conjecture}
%}}}~~~~~~~~~~~~~~~~~~~~~~~~~~~~~~~~~~~~~~~~~~~~~~~~~~~~~~~~~~~~~~~~~~~~~~~~~~~


%~~~~~~~~~~~~~~~~~~~~~~~~~~~~~~~~~~~~~~~~~~~~~~~~~~~~~~~~~~~~~~~~~~~~~~~~~~~~~~
% Aliases																{{{1
\newcommand{\infinity}{\infty}
%}}}~~~~~~~~~~~~~~~~~~~~~~~~~~~~~~~~~~~~~~~~~~~~~~~~~~~~~~~~~~~~~~~~~~~~~~~~~~~

%~~~~~~~~~~~~~~~~~~~~~~~~~~~~~~~~~~~~~~~~~~~~~~~~~~~~~~~~~~~~~~~~~~~~~~~~~~~~~~
% Missing unicode chars, other brokenness in ucs/inputenc {{{1
\DeclareUnicodeCharacter{183}{\cdot}						% ·
\DeclareUnicodeCharacter{931}{\ensuremath\Sigma}			% Σ
\DeclareUnicodeCharacter{9001}{\ensuremath\langle}			% 〈
\DeclareUnicodeCharacter{9002}{\ensuremath\rangle}			% 〉
\DeclareUnicodeCharacter{9608}{\ensuremath\blacksquare}		% █
\DeclareUnicodeCharacter{1013}{\in}							% ϵ
\DeclareUnicodeCharacter{8213}{---}							% ―

\renewcommand{\textcent}{\cent}
\renewcommand{\textcurrency}{\currency}
\renewcommand{\textyen}{\yen}
\renewcommand{\textbrokenbar}{\brokenvert}
%}}}~~~~~~~~~~~~~~~~~~~~~~~~~~~~~~~~~~~~~~~~~~~~~~~~~~~~~~~~~~~~~~~~~~~~~~~~~~~

\usepackage{framed}
% \newcounter{RQCounter}
% \newcommand{\RQ}[1]{%
% 	\stepcounter{RQCounter}
% 	\begin{framed}%
% 		\noindent\textbf{Research Question \arabic{RQCounter}: }%
% 		#1\end{framed}
% }

\newcounter{RQCounter}
\newcommand{\RQ}[1]{%
	\stepcounter{RQCounter}
		\textbf{\\RQ\arabic{RQCounter}: }%
		\textit{#1}
}

\newcommand{\cmark}{\ding{51}}%
\newcommand{\xmark}{\ding{55}}%

\DeclareTextFontCommand{\findings}{\normalfont\itshape\bfseries}

\newlength{\emstr}
\setlength{\emstr}{0.75em plus 1ex minus 1ex}
\newcommand{\boldpara}[1]{%
	\smallskip%
	\par\noindent\textbf{\textit{#1}}\hspace{\emstr}
}%

\def\sectionautorefname{Section}
\def\subsectionautorefname{Section}
\def\subsubsectionautorefname{Section}

% vim:foldmethod=marker

%%%%%%%%%%%%%%%%%%%%%%%%%%%%%%%%% Comments %%%%%%%%%%%%%%%%%%%%%%%%%%%%%
\newboolean{showcomments}
\setboolean{showcomments}{true} % comment this line to deactivate comments
\ifthenelse{\boolean{showcomments}}{
  \newcommand{\nbc}[3]{
    {\colorbox{#3}{\bfseries\sffamily\scriptsize\textcolor{white}{#1}}}%
    {\textcolor{#3}{\sf\small
    %$\blacktriangleright$
    \textit{#2}
    %$\blacktriangleleft$
    }}}
  \newcommand{\todo}[1]{\nbc{TODO}{#1}{blue}\xspace}
  \newcommand{\mar}[1]{\nbc{MAR}{#1}{cyan}\xspace}
  \newcommand{\earl}[1]{\nbc{EARL}{#1}{teal}\xspace}
  \newcommand{\fe}[1]{\nbc{FEDERICA}{#1}{violet}\xspace}
  \newcommand{\carlos}[1]{\nbc{CARLOS}{#1}{olive}\xspace}
}{
  \newcommand{\nbc}[3]{}
  \newcommand{\todo}[1]{}
  \newcommand{\mar}[1]{}
  \newcommand{\earl}[1]{}
  \newcommand{\fe}[1]{}
  \newcommand{\carlos}[1]{}
}

\newcommand{\sq}[1]{`{#1}'}